\documentclass[12pt]{article}
% \batchmode
\usepackage{url}
\urlstyle{sf}
\usepackage[svgnames]{xcolor}
\usepackage[colorlinks,linkcolor=Blue,citecolor=Blue,urlcolor=Blue]{hyperref}
\usepackage{amssymb}
\usepackage{amsmath}
\usepackage{framed}
\usepackage{mdframed}
% \usepackage{geometry}
% \usepackage{pdflscape}
\newmdenv[backgroundcolor=yellow]{alert}
\newmdenv[backgroundcolor=red]{note}
\hyphenation{Suite-Sparse}
\hyphenation{Graph-BLAS}
\hyphenation{Suite-Sparse-Graph-BLAS}

\newenvironment{packed_itemize}{
\begin{itemize}
  \setlength{\itemsep}{1pt}
  \setlength{\parskip}{0pt}
  \setlength{\parsep}{0pt}
}{\end{itemize}}

\title{Named and defined types and operators}

\author{Tim Davis \\
\small davis@tamu.edu, Texas A\&M University.}

%-------------------------------------------------------------------------------
\begin{document}
%-------------------------------------------------------------------------------
\maketitle

\begin{abstract}
GraphBLAS needs a mechanism for providing names (as strings) to its types
and operators both built-in and user-defined.  These strings can then be
returned by a method that queries the type of a matrix. Closely related is a
need to define types and operators using strings, so that these strings can
be interpretted by a JIT compiler.
\end{abstract}

\newpage
%-------------------------------------------------------------------------------
\section{The problems}
%-------------------------------------------------------------------------------

%-------------------------------------------------------------------------------
\subsection{Cannot query the type of a matrix}
%-------------------------------------------------------------------------------
\label{matrix_type_query}

There currently is a fundamental difficulty in writing algorithms using
GraphBLAS.  It is impossible to query the type of a \verb'GrB_Matrix',
\verb'GrB_Vector', or \verb'GrB_Scalar'.  This makes it hard to write an
algorithm that depends on the types of its inputs.

For example, consider a shortest-path algorithm.  It would be given an
adjacency matrix of any type, and it would like to use the operators and
semirings that are approriate for this type.
This is impossible with the v2.0 C API.

The problem is ``solved'' in LAGraph by including the matrix type in
the \verb'LAGraph_Graph' data structure:

{\footnotesize
\begin{verbatim}
    struct LAGraph_Graph_struct
    {
        GrB_Matrix   A;          // the adjacency matrix of the graph
        GrB_Type     A_type;     // the type of scalar stored in A
        ...
    } ;
\end{verbatim}}

Where \verb'A_type' is the type of the matrix \verb'A'.  However, this is
clumsy and error-prone.  It duplicates information that is already available
inside the matrix \verb'A', just inaccessible.  If the type of \verb'A' and
the variable \verb'A_type' do not match, bad things can happen inside the
LAGraph algorithms.

We thus need a mechanism where the user application can query a
\verb'GrB_Matrix',
\verb'GrB_Vector', or
\verb'GrB_Scalar' for its type.

The argument: ``The user should know what types its matrices are''
is not a good argument against the need for this type query.
The matrix dimension should be known to the user too, but the GraphBLAS
C API spec provides this information with \verb'GrB_Matrix_nrows' anyway.

Furthermore, the user might know the matrix type, but a library such
as LAGraph, that resides in the layer between the user application and
the underlying GraphBLAS library, may not know the type, as in the
shortest-path algorithm example.

Should a library such as LAGraph ask this of the end user? 

\begin{quote}
{\em Please pass me your matrix, so I can run some cool algorithm on it
for you.  Oh, and tell me the type of the matrix because I have no
clue and can't ask GraphBLAS to tell me.  By the way, don't get it wrong or
else I'll segfault when doing \verb'GrB_extractTuples' to an array whose size
depends on the size of this unqueriable type. Good luck.}
\end{quote}

Surely not.

%-------------------------------------------------------------------------------
\subsection{Cannot query the type of a serialized ``blob''}
%-------------------------------------------------------------------------------
\label{blob_type_query}

Related to Section~\ref{matrix_type_query} is the problem with the
serialization/deserialization of a \verb'GrB_Matrix' to/from a ``blob'' of
bytes (the spec doesn't call it a blob but that's not relevant for this
discussion).

The current methods in the v2.0 C API are asymmetric, and need to be so:

{\footnotesize
\begin{verbatim}
GrB_Info GrB_Matrix_serialize       // serialize a GrB_Matrix to a blob
(
    // output:
    void *blob,                     // the blob, already allocated in input
    // input/output:
    GrB_Index *blob_size_handle,    // size of the blob on input.  On output,
                                    // the # of bytes used in the blob.
    // input:
    GrB_Matrix A                    // matrix to serialize
) ;

GrB_Info GrB_Matrix_deserialize     // deserialize blob into a GrB_Matrix
(
    // output:
    GrB_Matrix *C,      // output matrix created from the blob
    // input:
    GrB_Type type,      // type of the matrix C
    const void *blob,       // the blob
    GrB_Index blob_size     // size of the blob
) ;
\end{verbatim}}

When a matrix is serialized, the underlying library already knows its
\verb'GrB_Type', so this does not appear in the signature for
the \verb'GrB_Matrix_serialize' method.

Built-in types can be encoded into the \verb'blob', and this could be
reconstructed when the matrix is deserialized.
User-defined types are difficult.  All the GraphBLAS library knows about
a user-defined type is its size in bytes.  Even though the library knows the
matrix has a specific user-defined \verb'GrB_Type', the library cannot save
this information to the \verb'blob' directly.  The \verb'GrB_Type' is an
ephemeral pointer, and it goes away when the process finishes.

The purpose of serialization/deserialization is to create a blob of bytes that
can be saved to a file, sent to another MPI process  running the same GraphBLAS
library, or some other method that goes outside the scope of the process
creating the blob.  The \verb'GrB_Type' for user-defined types is just a
pointer to a \verb'malloc'ed block.  It cannot be sent to another process and
cannot be saved to a file.

In SuiteSparse:GraphBLAS, I extend the usage of \verb'GrB_Matrix_deserialize'
slightly, to make it easier to deserialize a matrix of built-in type.
I state in my user guide the \verb'GrB_Type type' input to 
\verb'GrB_Matrix_deserialize':
    {\em Required if the blob holds a matrix of user-defined type.  May be NULL
    if blob holds a built-in type; otherwise must match the type of C.}

However, this does not solve the problem for user-defined types.

Suppose a user writes a blob to a file with a user-defined type:

{\footnotesize
\begin{verbatim}
    typedef struct
    {
        float stuff [4][4] ;
        char whatstuff [64] ;
    }
    wildtype ;                      // C version of wildtype

    GrB_Type WildType ;             // GraphBLAS version of wildtype
    ...
    GrB_Type_new (&WildType, sizeof (wildtype)) ;
    GrB_Matrix_new (&A, WildType, ...) ;
    ...
    GrB_Matrix_serialize (blob, &blobsize, A) ;
    FILE *f = fopen ("mystuff.bin", "w") ; 
    fwrite (&blobsize, sizeof (GrB_Index), 1, f) ;
    fwrite (blob, sizeof (char), blobsize, f) ;
    fclose (f) ;
\end{verbatim}}

This is not a good use of \verb'GrB_Matrix_serialize'.  When the file is
reopened, the user must magically know that the blob contains a matrix
of \verb'GrB_Type WildType'.  How can it know this?  It's not information
that can be held in the blob itself.

{\footnotesize
\begin{verbatim}
    FILE *f = fopen ("mystuff.bin", "r") ; 
    GrB_Index blobsize ;
    fread (&blobsize, sizeof (GrB_Index), 1, f) ;
    void *blob = malloc (blobsize) ;
    fread (blob, sizeof (char), blobsize, f) ;
    fclose (f) ;
    GrB_Matrix_deserialize (&A, WildType, blob, blobsize) ;
\end{verbatim}}

This solution is fragile.  What if the wrong type is passed to deserialize?
Worse, what if the type has the wrong size?  A segfault or security problem
is waiting to happen, particularly if the file can from another source.

Yet the GraphBLAS library can provide little help here.  The library does not
know the name of the type (neither the \verb'typedef struct' name,
\verb'wildtype', nor the \verb'GrB_Type WildType').  All it knows is the size,
in this case probably 128 bytes.  The library could save in the blob some
indicator that the type is user-defined, of size 128 bytes.  However, the
library has no way of communicating this information to the user application.

%-------------------------------------------------------------------------------
\section{Difficulties in the solution}
%-------------------------------------------------------------------------------

Consider the simple case of querying a \verb'GrB_Matrix A' for its type.
Should this return a pointer to a \verb'GrB_Type t'?  If so, how long does this
variable \verb't' last?  And who owns it; is it part of the matrix
\verb'A'?  Or is it owned by the user application, who must at some point
free the variable \verb't'?

Consider the deserialization case.  Querying the blob for its type and
returning the type as a \verb'GrB_Type t' is not a good solution.  That is
possible if the type is built-in, but not possible for user-defined types.
The GraphBLAS library has no knowledge of the user-defined type except
the pointer itself (which cannot be saved in the blob) and its size in bytes,
which would be helpful to know but is not sufficient to know the actual
type: \verb'WildType', constructed from the \verb'typedef wildtype'.

Furthermore, type information cannot be passed between MPI processes.
One process cannot send a \verb'GrB_Type' pointer to another process.

As a result of all these issues, return a \verb'GrB_Type' for type queries
is not a good idea.

%-------------------------------------------------------------------------------
\section{The solution: named types (using strings)}
%-------------------------------------------------------------------------------

{\bf Give each type a name as a string, including built-in and user-defined,
and allow the type of a matrix to be querried by returning this string.}

%-------------------------------------------------------------------------------
\subsection{A revised {\tt GrB\_Type\_new} method}
%-------------------------------------------------------------------------------
\label{gxb_type_new}

The name of a built-in type should match the underlying C type, but held
as a string.  For example, the name of the \verb'GrB_BOOL' type should be
\verb'"bool"', \verb'GrB_INT32' is \verb'"int32_t"', and \verb'GrB_FP32' is
\verb'"float"', and so on.  All types are named with strings.

Furthermore, the GraphBLAS library should know more about user-defined
types than just their name (as a string) and size (in bytes).  It should
know the entire \verb'typedef', as a string, so it can employ JIT acceleration
for user-defined types.

SuiteSparse:GraphBLAS includes the following \verb'GxB' extension:

{\footnotesize
\begin{verbatim}
// GxB_Type_new creates a type with a name and definition that are known to
// GraphBLAS, as strings.  The type_name is any valid string (max length of 128
// characters, including the required null-terminating character) that may
// appear as the name of a C type created by a C "typedef" statement.  It must
// not contain any white-space characters.  Example, creating a type of size
// 16*4+4 = 68 bytes, with a 4-by-4 dense float array and a 32-bit integer:
//
//      typedef struct { float x [4][4] ; int color ; } myquaternion ;
//      GrB_Type MyQtype ;
//      GxB_Type_new (&MyQtype, sizeof (myquaternion), "myquaternion",
//          "typedef struct { float x [4][4] ; int color ; } myquaternion ;") ;
//
// The type_name and type_defn are both null-terminated strings.  Currently,
// type_defn is unused, but it will be required for best performance when a JIT
// is implemented in SuiteSparse:GraphBLAS (both on the CPU and GPU).  User
// defined types created by GrB_Type_new will not work with a JIT.
//
// At most GxB_MAX_NAME_LEN characters are accessed in type_name; characters
// beyond that limit are silently ignored.

#define GxB_MAX_NAME_LEN 128

GrB_Info GxB_Type_new           // create a new named GraphBLAS type
(
    GrB_Type *type,             // handle of user type to create
    size_t sizeof_ctype,        // size = sizeof (ctype) of the C type
    const char *type_name,      // name of the type (max 128 characters)
    const char *type_defn       // typedef for the type (no max length)
) ;
\end{verbatim}}

The only downside to this solution is that the \verb'typedef' is repeated
twice: once in the C program to define the type \verb'myquaterion', in this
exmpale, and again in the string passed to \verb'GxB_Type_new'.  This
duplication could be avoided with C macros.

%-------------------------------------------------------------------------------
\subsection{A new {\tt GrB\_Matrix\_type\_name} method}
%-------------------------------------------------------------------------------

Since all types, including built-in types, are given a name, and since this
name does not exceed 128 bytes in length, the user can query the type of
a matrix with the following method:

{\footnotesize
\begin{verbatim}
GrB_Info GxB_Matrix_type_name      // return the name of the type of a matrix
(
    char *type_name,        // name of the type (char array of size at least
                            // GxB_MAX_NAME_LEN, owned by the user application).
    const GrB_Matrix A      // matrix to query
) ;
\end{verbatim}}

Usage:
{\footnotesize
\begin{verbatim}
    char typename [GxB_MAX_NAME_LEN] ;
    GxB_Matrix_type_name (typename, A) ;
\end{verbatim}}

If the matrix has type \verb'GrB_FP64', the \verb'typename' now contains the
null-terminated string \verb'"double"'.  If the matrix has type \verb'WildType'
or \verb'MyQtype', then the \verb'typename' array holds the null-terminated
string \verb'wildtype' or \verb'myquaternion', respectively.

Ownership and lifetime of the return value from \verb'GxB_Matrix_type_name'
is well-defined.  The result is copied into a user-owned \verb'char' array.
The user application has full control and ownership of this array, and the
array \verb'typename' can outlive the existence of the matrix \verb'A'
and/or its type, \verb'WildType'.

%-------------------------------------------------------------------------------
\subsection{A new {\tt GrB\_Type\_size} method}
%-------------------------------------------------------------------------------

Methods such as \verb'GrB_Matrix_extractTuples' require a user array,
call it \verb'X' passed to it.  This array must be able to hold
\verb'nvals' entries, each of size equal to the number of bytes needed
to hold the size of the matrix type.

\verb'GrB_Matrix_nvals' can report the number of values.

What is the type of the matrix?  And what is the size of this type, so
the user can allocate the proper sized array?  GraphBLAS cannot tell you,
with the current v2.0 C API.  There is a need for querying the size of
a type, so that \verb'GxB_Type_size (&s, GrB_FP32)' would return
\verb'sizeof(float)', or typically 4, for example.

{\footnotesize
\begin{verbatim}
    GrB_Info GxB_Type_size          // determine the size of the type
    (
        size_t *size,               // the sizeof the type
        const GrB_Type type         // type to determine the sizeof
    ) ;
\end{verbatim}}

%-------------------------------------------------------------------------------
\subsection{A new {\tt GrB\_Matrix\_serialized\_type\_name} method}
%-------------------------------------------------------------------------------

Now that the type has a name, as a string, the string can be safely
inserted by the library into a serialized blob.  Then, when the blob is
reloaded from a file, or received from another MPI process, the string
can be safely parsed and returned to the user application:

{\footnotesize
\begin{verbatim}
GrB_Info GxB_deserialize_type_name  // return the type name of a blob
(
    // output:
    char *type_name,        // name of the type (char array of size at least
                            // GxB_MAX_NAME_LEN, owned by the user application).
    // input, not modified:
    const void *blob,       // the blob
    GrB_Index blob_size     // size of the blob
) ;
\end{verbatim}}

Usage: reading a blob from a file and getting its type

{\footnotesize
\begin{verbatim}
    FILE *f = fopen ("mystuff.bin", "r") ; 
    GrB_Index blobsize ;
    fread (&blobsize, sizeof (GrB_Index), 1, f) ;
    void *blob = malloc (blobsize) ;
    fread (blob, sizeof (char), blobsize, f) ;
    fclose (f) ;

    // get the typename of the matrix held in the blob:
    char typename [GxB_MAX_NAME_LEN] ;
    GxB_deserialized_type_name (typename, blob, blob_size) ;

    if (strcmp (typename, "wildtype"))
    {
        GrB_Matrix_deserialize (&A, WildType, blob, blobsize) ;
    }
    else if (strcmp (typename, "myquaternion"))
    {
        GrB_Matrix_deserialize (&A, MyQType, blob, blobsize) ;
    }
    else if (strcmp (typename, "float"))
    {
        GrB_Matrix_deserialize (&A, GrB_FP32, blob, blobsize) ;
    }
        ... and so on ...  \end{verbatim}}

Better yet, if the blob has a built-in type, the library can handle this
itself.  This can work for a blob of a matrix of any built-in type:

{\footnotesize
\begin{verbatim}
    GrB_Matrix_deserialize (&A, NULL, blob, blobsize) ;
\end{verbatim}}

Then the user application need only compare the \verb'typename' string
for its known user-defined types, and after that, assume the blob has
a built-in type.

Since the library now knows the name of the type of the matrix held in the
blob, it can know if the name is a recognized built-in type (
\verb'"bool"',
\verb'"int8_t"',
\verb'"float"', ...
\verb'"double"').  If the \verb'type' input parameter to \newline
\verb'GrB_Matrix_deserialize' is \verb'NULL', and the blob has a known
built-in type, it can handle this itself.  The library can also record the
name and size in bytes of a user-defined type, and if the \verb'type'
parameter is \verb'NULL', it can refuse to deserialize it, since it nows
the blob has a matrix of user-defined type.

%-------------------------------------------------------------------------------
\subsection{A new method to convert the string to the type}
%-------------------------------------------------------------------------------

Eventually, there would be a need to convert the \verb'typename' string
to an actual \verb'GrB_Type'.  This can easily be done by the GraphBLAS
library for built-in types.  For user-defined types, I suggest that this
function return \verb'GrB_NO_VALUE' and a \verb'NULL' type, so the user
application can convert the string to a type.

{\footnotesize
\begin{verbatim}
GrB_Info GxB_Type_from_name     // return the built-in GrB_Type from a name
(
    GrB_Type *type,             // built-in type, or NULL if user-defined
    const char *type_name       // array of size at least GxB_MAX_NAME_LEN
) ;
\end{verbatim}}

Usage:

{\footnotesize
\begin{verbatim}
    // to query the type of A:
    size_t typesize ;
    GxB_Type_size (&typesize, type) ;       // works for any type
    GrB_Type atype ;
    char atype_name [GxB_MAX_NAME_LEN] ;
    GxB_Matrix_type_name (atype_name, A) ;
    GxB_Type_from_name (&atype, atype_name) ;
    if (atype == NULL)
    {
        // This is not yet an error.  It means that A has a user-defined type.
        if ((strcmp (atype_name, "myfirsttype")) == 0) atype = MyType1 ;
        else if ((strcmp (atype_name, "myquaternion")) == 0) atype = MyQType ;
        else if ((strcmp (atype_name, "wildetype")) == 0) atype = WildType ;
        else { ... this is now an error ... the type of A is unknown.  }
        }
    }
\end{verbatim}}

%-------------------------------------------------------------------------------
\section{JIT acceleration}
%-------------------------------------------------------------------------------

%-------------------------------------------------------------------------------
\subsection{The problem: JIT acceleration is impossible with the current API}
%-------------------------------------------------------------------------------

With types named with a string as \verb'myquaternion' for example, and
also defined with a string as this for example:

{\footnotesize
\begin{verbatim}
    "typedef struct { float x [4][4] ; int color ; } myquaternion ;"
\end{verbatim}}

\noindent
the GraphBLAS library has powerful tools to reason about the types of
its matrices.  It can provide this information back to the user application,
through safe type queries.  It can use this information to make
serialization/deserialization safer and more reliable.  It can allow
LAGraph to ditch its clumsy solution of passing along the type of a matrix
with the matrix itself.

However, this is only a partial solution in how a GraphBLAS library could deal
with user-defined types.  These UDT's also require operators.  Currently, all a
GraphBLAS library knows about an operator is a function pointer.  It doesn't
even know the name of these operators, just a pointer.  This limits the
performance of a GraphBLAS library when working on user-defined types.

The GraphBLAS library may wish to construct fast kernels at run time for
user-defined types and operators.  This requies the library to open a new
file, write a C program into it (with these user-defined types and operators),
compile the function, link it in, and then call the kernel.
The next time this kernel is called, it can keep a record of it, and
not reconstruct and recompile it but just use the existing linked-in
compiled code.

This is essential on the GPU, since the GPU cannot call a function pointer
defined on the CPU and passed to \verb'GrB_BinaryOp_new'.

%-------------------------------------------------------------------------------
\subsection{The solution: give names and definitions to all user-defined
operators}
%-------------------------------------------------------------------------------

Consider the following extension:

{\footnotesize
\begin{verbatim}
GrB_Info GxB_BinaryOp_new
(
    GrB_BinaryOp *op,               // handle for the new binary operator
    GxB_binary_function function,   // pointer to the binary function
    GrB_Type ztype,                 // type of output z
    GrB_Type xtype,                 // type of input x
    GrB_Type ytype,                 // type of input y
    const char *binop_name,         // name of the user function
    const char *binop_defn          // definition of the user function
) ;
\end{verbatim}}

This method augments \verb'GrB_BinaryOp_new' by giving the new operator
a name, as a string (of length \verb'GxB_MAX_NAME_LEN') and by giving the
definition of the entire function itself.  For example, suppose the
user wanted to define a \verb'mod' operator that worked on \verb'uint64_t'
types.  The current solution is limited:

{\footnotesize
\begin{verbatim}
    void mod_function (void *z, const void *x, const void *y)
    {
        uint64_t a = (*((uint64_t *) x)) ;
        uint64_t b = (*((uint64_t *) y)) ;
        (*((uint64_t *) z)) = a % b ;
    }
    ...
    GrB_BinaryOp Mod ;
    GrB_BinaryOp_new (&Mod, mod_function, GrB_UINT64, GrB_UINT64, GrB_UINT64)) ;
\end{verbatim}}

Consider the information that the GraphBLAS library knows about this
new operator:  it has an ephemeral pointer, \verb'Mod', to some
\verb'malloc'd block of memory that holds the \verb'GrB_BinaryOp' 
object.  This object has three pointers to three \verb'GrB_Type' objects.
These are built-in types, so the library knows that they correspond
to the C \verb'uint64_t' type.  If the type was user-defined, unless the
\verb'GxB_Type_new' is used (Section \ref{gxb_type_new}), all it knows
is the size of these user-defined types.

The library has no idea of the contents of the \verb'mod_function'.  This
function pointer is compiled for use on the CPU.  It cannot be called from the
GPU, so GPU acceleration of this user-defined function is impossible.

Now consider a better alternative:

{\footnotesize
\begin{verbatim}
    GxB_BinaryOp_new (&Mod, mod_function, GrB_UINT64, GrB_UINT64, GrB_UINT64
        "mod_function",
        "void mod_function (uint64_t *z, const uint64_t *x, const uint64_t *y)"
        "{"
        "    z = x % y ;"
        "}") ;
\end{verbatim}}

First, the parameters no longer need to be \verb'void' and require typecasting.
Typecasting a pointer from \verb'void *' to \verb'uint64_t *' requires no work
at run time, just some work by the compiler, but it makes the code harder to
read.  Now, however, the library knows that the types have the C name of
\verb"uint64_t" so it can use those directly.

For a JIT on the CPU, the GraphBLAS library could create a short
file like this.  The library it knows the name (\verb'"mod_function"')
and so it can build a full name of a kernel, and it knows the definition
of the \verb'mod_function' itself, as a string, so it can build a
static inline copy of it:

{\footnotesize
\begin{verbatim}
    static inline
        void mod_function (uint64_t *z, const uint64_t *x, const uint64_t *y)
        {
            z = x % y ;
        }
    #define OP(z,x,y) mod_function (&z, &x, &y)
    #define GB_EWISEADD_KERNEL_NAME GB_ewiseAdd_kernel_for_mod_function
    #include "GB_eWiseAdd_kernel.c"
\end{verbatim}}

where the file \verb'GB_eWiseAdd_kernel.c' would be part of the library,
like my \verb'Source/Template/*.c' files in SuiteSparse:GraphBLAS.  Those
kernels work with any data type and any operator, where the types and operators
are \verb'#define'd by C macros.

I have not implemented this, but it is feasible.  I have implemented
kernels like this for the GPU, but only for built-in types and operators.

This kernel would be far faster than computing the ewiseadd operation
by calling the function pointer every time.  Function pointers are very slow.

%-------------------------------------------------------------------------------
\subsection{Named and defined monoids and semirings}
%-------------------------------------------------------------------------------

Named monoids are not essential, since their name can be constructed from
the name of the underlying binary operator, whether user-defined or built-in.
However, the monoid identity value is likely needed as a string, since the
GraphBLAS library would otherwise have difficulty in constructing a string
that defines it an placing that string in a file to be compiled.

A semiring is even simpler.  It consists of just a binary multiplicative
operator, and a monoid.  The name of the semiring can be built from the
names of these two objects, as well as the names of the types they operate
on (up to 3 of them for the multiplier and monoid, plus 3 more if the types
of the 3 matrices are also considered).  All of these objects would have
named and defined types and operators that underly them, so an entire
semiring could be encoded in a string, placed in a file, compiled, linked,
and executed.

%-------------------------------------------------------------------------------
\section{Summary}
%-------------------------------------------------------------------------------

Use strings to give all types and operators a name.  The name should be
short, of some fixed maximum length.

Use the strings to allow all types to be queried, for all objects:

    \begin{enumerate}
    \item What is the type of a \verb'GrB_Matrix'?
    \item What is the type of a \verb'GrB_Vector'?
    \item What is the type of a \verb'GrB_Scalar'?
    \item What is the type of a matrix held inside a serialized blob?
    \item What is the type of the $y$ input to a binary operator $f=(x,y)$?
    \item etc.
    \end{enumerate}

Use strings to give all types and operators a precise definition.  The
definition cannot be limited by a maximum length.   These definitions can then
be used by a JIT on both the CPU and the GPU, so that user-defined types
and operators can be just as fast as built-in ones.

\end{document}

